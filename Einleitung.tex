\section{Einleitung}
%
Auf dem Computer werden jegliche Informationen in Form von Bits, also Zahlenfolgen bestehend aus 0 und 1, gespeichert. Der Speicherplatz ist dabei beschränkt und es ist oft erstrebenswert die Menge an Bits, welche eine Information repräsentieren, zu reduzieren. Man unterscheidet dabei zwischen zwei Arten dieser sogenannten Kompression:\\
Bei der verlustfreien Kompression wird versucht die Entropie der Zahlenfolge, welche die Information repräsentiert, zu maximieren, um so den selben Informationsgehalt auf weniger Bits zu verteilen.
Höhere Kompressionsgrade erhält man hingegen mit verlustbehafteter Kompression. Dabei wird das Informationsgut in Klassen verschiedener Wichtigkeit aufgeteilt, um so von einem bestimmten Grad an Details -- im Tausch gegen reduzierten Speicherbedarf -- absehen zu können. Eine Methode zur Aufteilung der Informationen in Detailstufen ist die Wavelet-Transformation. Neben der Kompression kommt sie auch bei graphischen Anwendungen zum Einsatz.\\
Es sollen hier nun das mathematische Grundgerüst der Wavelet-Transformation, die Multiskalenanalyse, sowie die Bildkompression am Beispiel von Haar-Wavelets, und graphische Anwendungen am Beispiel von B-Splines vorgestellt werden.