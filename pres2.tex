\section{Algorithmen} {

\begin{frame}{Algorithmen}
\begin{center}
\begin{tabular}{ccc}
\begin{tabular}{c}Haar-Wavelet- \\ Transformation\end{tabular} & 
$\longleftrightarrow$ & 
\begin{tabular}{c}Basiswechsel zwischen \\ Orthonormalbasen\end{tabular} \\
\end{tabular}
\\[1.0cm]
\end{center}

Zwei Ansätze:
\begin{itemize}
\item einzelne Koeffizienten: Projektion durch Skalarprodukt
\item gesamte Transformation: Filterbank-Algorithmus
\end{itemize}

\end{frame}


\begin{frame}{Erinnerung: Filterbank}

Die Filterbank-Transformation:\\[-0.1cm]
\[
\begin{array}{c c c c c c c c c c c}
C^{k} & \rightarrow & C^{k-1} & \rightarrow & C^{k-2} & \rightarrow & \dots & \rightarrow & C^{1} & \rightarrow & \mathbf{C^{0}} \\
 & \searrow & & \searrow & & \searrow & & \searrow & & \searrow & \\
 & & \mathbf{D^{k-1}} & & \mathbf{D^{k-2}} & & \dots & & \mathbf{D^{1}} & & \mathbf{D^{0}} \\
\end{array}
\]
\\[0.5cm]
mit
\\[0.1cm]
\center $C^{j} \quad = \quad P^{j} \enspace C^{j-1} \quad + \quad Q^{j} \enspace D^{j-1}$
\\
bzw.
\\
$C^{j-1} \quad = \quad A^{j} \enspace C^{j} \qquad, \qquad D^{j-1} \quad = \quad B^{j} \enspace C^{j}$
\\[0.25cm]

\pause[2] \textbf{Mehrere Matrixmultiplikationen -- Teuer!?}
\end{frame}

% Hier wird darauf hingewiesen, dass P und Q im Fall der Haar-Wavelet-Trafo fast nur Nullen enthalten, die Filterbank-Schritte also schneller direkt Implementiert werden können.

\begin{frame}{1D Haar-Wavelet-Transformation}
Einzelschritt im Speicher:
\\[1.0cm]
\begin{adjustbox}{max width=\textwidth ,center}
\begin{tabular}{c}
\settowidth\mylen{$\phi_{2i+1}^{j}$}
$
\begin{array}{c C{\mylen} C{\mylen} C{1.5\mylen} C{\mylen} C{\mylen} C{1.5\mylen} C{\mylen} C{\mylen}}
&
\phi_{0}^{j} & \phi_{1}^{j} & ... & \phi_{2i}^{j} & \phi_{2i+1}^{j} & ... & \phi_{n-2}^{j} & \phi_{n-1}^{j} \\ \cline{2-9}
C \hphantom{'}:
&
\multicolumn{1}{|c}{} & 
\multicolumn{1}{c}{} & 
\multicolumn{1}{c}{} & 
\multicolumn{1}{|c|}{a} & 
\multicolumn{1}{|c|}{b} & 
\multicolumn{1}{c}{} & 
\multicolumn{1}{c}{} & 
\multicolumn{1}{c|}{}
\\ \cline{2-9}
\end{array}
$
\\[-0.3cm]

\settowidth\mylen{$\phi_{2i+1}^{j}$}
$
\begin{array}{c C{\mylen} C{\mylen} C{1.4\mylen} C{\mylen} C{\mylen} C{1.4\mylen} C{\mylen} C{\mylen}}
 & & & & & & & & \\
 \hphantom{C':} &  &  & \multicolumn{2}{c}{\swarrow} & \multicolumn{2}{c}{\searrow} &  &  \\
\end{array}
$
\\[0.5cm]

\settowidth\mylen{$\psi_{\frac{n}{2}-1}^{j-1}$}
$
\begin{array}{c C{\mylen} C{0.5\mylen} C{\mylen} C{0.5\mylen} C{\mylen} C{\mylen} C{0.5\mylen} C{\mylen} C{0.5\mylen} C{\mylen}}
&
\phi_{0}^{j-1} &
... &
\phi_{i}^{j-1} &
... &
\phi_{\frac{n}{2}-1}^{j-1} &
\psi_{0}^{j-1} &
... &
\psi_{i}^{j-1} &
... &
\psi_{\frac{n}{2}-1}^{j} \\ \cline{2-11}
C':
&
\multicolumn{1}{|c}{} & 
\multicolumn{1}{c}{} & 
\multicolumn{1}{|c|}{\frac{a+b}{\sqrt{2}}} & 
\multicolumn{1}{c}{} & 
\multicolumn{1}{c:}{} & 
\multicolumn{1}{:c}{} & 
\multicolumn{1}{c}{} & 
\multicolumn{1}{|c|}{\frac{a-b}{\sqrt{2}}} & 
\multicolumn{1}{c}{} & 
\multicolumn{1}{c|}{}
\\ \cline{2-11}
\end{array}
$
\end{tabular}
\end{adjustbox}
\end{frame}

% An dieser Stelle wir der Analyse-Schritt-Algorithmus an der Tafel aufgeschrieben.

\begin{frame}{1D Haar-Wavelet-Transformation}
\center
Gesamt-Transformation im Speicher:
\\[-0.75cm]
\[
\begin{array}{*{8}{C{0.9cm}}}
\multicolumn{8}{c}{\vdots} \\
\hline \multicolumn{8}{|c|}{C^3} \\ \hline
\multicolumn{4}{r}{\swarrow} & \multicolumn{4}{l}{\searrow}\\
\hline \multicolumn{4}{|c|}{C^2} & \multicolumn{4}{|c|}{D^2}\\ \hline
\multicolumn{2}{r}{\swarrow} & \multicolumn{2}{l}{\searrow}\\
\cline{1-4} \multicolumn{2}{|c|}{C^1} & \multicolumn{2}{|c|}{D^1}\\ \cline{1-4}
\multicolumn{1}{r}{\swarrow} & \multicolumn{1}{l}{\searrow}\\
\cline{1-2} \multicolumn{1}{|c|}{C^0} & \multicolumn{1}{|c|}{D^0}\\ \cline{1-2}
\\ \hline
 & & & & & & & \\
\hline \multicolumn{1}{|c|}{C^0} & \multicolumn{1}{|c|}{D^0} & \multicolumn{2}{|c|}{D^1} & \multicolumn{4}{|c|}{D^2}\\ \hline
\end{array}
\]

\end{frame}

\begin{frame}[fragile]{1D Haar-Wavelet-Transformation}
\begin{adjustbox}{width=0.4\textwidth, center}
$
\begin{array}{*{8}{C{0.9cm}}}
\multicolumn{8}{c}{\vdots} \\
\hline \multicolumn{8}{|c|}{C^3} \\ \hline
\multicolumn{4}{r}{\swarrow} & \multicolumn{4}{l}{\searrow}\\
\hline \multicolumn{4}{|c|}{C^2} & \multicolumn{4}{|c|}{D^2}\\ \hline
\multicolumn{2}{r}{\swarrow} & \multicolumn{2}{l}{\searrow}\\
\cline{1-4} \multicolumn{2}{|c|}{C^1} & \multicolumn{2}{|c|}{D^1}\\ \cline{1-4}
\multicolumn{1}{r}{\swarrow} & \multicolumn{1}{l}{\searrow}\\
\cline{1-2} \multicolumn{1}{|c|}{C^0} & \multicolumn{1}{|c|}{D^0}\\ \cline{1-2}
\\ \hline
 & & & & & & & \\
\hline \multicolumn{1}{|c|}{C^0} & \multicolumn{1}{|c|}{D^0} & \multicolumn{2}{|c|}{D^1} & \multicolumn{4}{|c|}{D^2}\\ \hline
\end{array}
$
\end{adjustbox}
\center
\begin{adjustbox}{minipage=0.95\linewidth, fbox}
\begin{verbatim}
def analysis(C):
    C' = copy(C)    # <-- optional
    h = C.size
    while h > 1:
        C'[0:h] = analysisStep(C'[0:h])
        h = h // 2
    return C'
\end{verbatim}
\end{adjustbox}

\end{frame}

% An dieser Stelle mache ich an der Tafel die Laufzeit-Analyse. Dann kommt die Überleitung zu 2D-Signalen

\begin{frame}{2D Haar-Wavelet-Transformationen}
\textbf{Ergebnis:} \\
Die Algorithmen aus dem 1D-Fall können zur effizienten Implementierung der Transformation im Standard- und im Nicht-Standard-Fall wiederverwendet werden.
\\[1.0cm]
\textbf{Laufzeitkomplexität} für $n \times n$-Input:
\begin{itemize}
\item Standard-Basis: \\
$4n^2 - 4n$
\item Nicht-Standard-Basis: \\
$\frac{8}{3}(n^2-1)$
\end{itemize}

\end{frame}

% Das dauert deutlich zu lange....
%\begin{frame}{2D Standard-Haar-Wavelet-Transformation}
%\begin{center}
%\textbf{2D Standard-Basis}
%\end{center}
%~\\[-1.0cm]
%Basisfunktionen sind Tensorprodukte der 1D Basisfunktionen.\\
%\[\mathbf{v} = \sum_{k,l=0}^{n-1} a_{k,l}\phi_k^j(x)\phi_l^j(y)\]
%\\[-0.1cm]
%
%Ziel:\\[0.5cm]
%
%\begin{adjustbox}{max width=\textwidth ,center}
%\begin{tabular}{c c c}
%
%$
%\begin{array}{c|c|c|c|c}
% \multicolumn{1}{c}{}& \multicolumn{1}{c}{\phi_{0}^{j}(x)} & \multicolumn{1}{c}{\phi_{1}^{j}(x)} & \multicolumn{1}{c}{\phi_{2}^{j}(x)} & ... \\ \cline{2-5}
% \phi_{0}^{j}(y) & a_{0,0} & a_{1,0} & a_{2,0} & \\ \cline{2-5}
% \phi_{1}^{j}(y) & a_{0,1} & a_{1,1} & a_{2,1} & \\ \cline{2-5}
% \phi_{2}^{j}(y) & a_{0,2} & a_{1,2} & a_{2,2} & \\ \cline{2-5}
% \vdots & & & & 
%\end{array}
%$
%&
%$\overset{?}{\rightarrow}$
%&
%$
%\begin{array}{c|c|c|c|c}
% \multicolumn{1}{c}{}& \multicolumn{1}{c}{\phi_{0}^{0}(x)} & \multicolumn{1}{c}{\psi_{0}^{0}(x)} & \multicolumn{1}{c}{\psi_{0}^{1}(x)} & ... \\ \cline{2-5}
% \phi_{0}^{0}(y) & b_{0,0} & b_{1,0} & b_{2,0} & \\ \cline{2-5}
% \psi_{0}^{0}(y) & b_{0,1} & b_{1,1} & b_{2,1} & \\ \cline{2-5}
% \psi_{0}^{1}(y) & b_{0,2} & b_{1,2} & b_{2,2} & \\ \cline{2-5}
% \vdots & & & & 
%\end{array}
%$ 
%\\
%
%\end{tabular}
%\end{adjustbox}
%\end{frame}
%
%% An dieser Stelle führe an der Tafel den Ausklammer-Trick vor...
%
%\begin{frame}{2D Standard-Haar-Wavelet-Transformation}
%Anwenden der 1D Transformation auf Zeilen und Spalten:
%\\[0.75cm]
%\begin{adjustbox}{max width=\textwidth ,center}
%\begin{tabular}{c c c}
%$
%\begin{array}{c|c|c|c|c}
% \multicolumn{1}{c}{}& \multicolumn{1}{c}{\phi_{0}^{j}(x)} & \multicolumn{1}{c}{\phi_{1}^{j}(x)} & \multicolumn{1}{c}{\phi_{2}^{j}(x)} & ... \\ \cline{2-5}
% \phi_{l}^{j}(y)  & a_{0,l} & a_{1,l} & a_{2,l} & \\ \cline{2-5}
%\end{array}
%$
%&
%$\begin{array}{c}
%\\ \overset{1D}{\longleftrightarrow} \\
%\end{array}$
%&
%$
%\begin{array}{c|c|c|c|c}
% \multicolumn{1}{c}{}& \multicolumn{1}{c}{\phi_{0}^{0}(x)} & \multicolumn{1}{c}{\psi_{0}^{0}(x)} & \multicolumn{1}{c}{\psi_{0}^{1}(x)} & ... \\ \cline{2-5}
% \phi_{l}^{j}(y) & a_{0,l}' & a_{1,l}' & a_{2,l}' & \\ \cline{2-5}
%\end{array}
%$ 
%\\
%\\ \hline
%\\
%$
%\begin{array}{c|c|}
% \multicolumn{1}{c}{}& \multicolumn{1}{c}{\psi_{k}^{i}(x)} \\ \cline{2-2}
% \phi_{0}^{j}(y) & a_{k,0}' \\ \cline{2-2}
% \phi_{1}^{j}(y) & a_{k,1}' \\ \cline{2-2}
% \phi_{2}^{j}(y) & a_{k,2}' \\ \cline{2-2}
% \vdots & 
%\end{array}
%$
%&
%$\overset{1D}{\longleftrightarrow}$
%&
%$
%\begin{array}{c|c|}
% \multicolumn{1}{c}{}& \multicolumn{1}{c}{\psi_{k}^{i}(x)} \\ \cline{2-2}
% \phi_{0}^{0}(y) & b_{k,0} \\ \cline{2-2}
% \psi_{0}^{1}(y) & b_{k,1} \\ \cline{2-2}
% \psi_{0}^{1}(y) & b_{k,2} \\ \cline{2-2}
% \vdots & 
%\end{array}
%$ 
%\end{tabular}
%\end{adjustbox}
%\end{frame}
%
%\begin{frame}{2D Standard-Haar-Wavelet-Transformation}
%2D Standard-Transformation durch 1D Transformationen:
%\\[0.75cm]
%\begin{adjustbox}{max width=\textwidth ,center}
%\begin{tabular}{c c c}
%
%$
%\begin{array}{c|c|c|c|c}
% \multicolumn{1}{c}{}& \multicolumn{1}{c}{\phi_{0}^{j}(x)} & \multicolumn{1}{c}{\phi_{1}^{j}(x)} & \multicolumn{1}{c}{\phi_{2}^{j}(x)} & ... \\ \cline{2-5}
% \phi_{0}^{j}(y) & a_{0,0} & a_{1,0} & a_{2,0} & \\ \cline{2-5}
% \phi_{1}^{j}(y) & a_{0,1} & a_{1,1} & a_{2,1} & \\ \cline{2-5}
% \phi_{2}^{j}(y) & a_{0,2} & a_{1,2} & a_{2,2} & \\ \cline{2-5}
% \vdots & & & & 
%\end{array}
%$
%&
%$
%\begin{array}{c}
%\\
%\overset{\mbox{\footnotesize 1D}}{\longrightarrow} \\
%\overset{\mbox{\footnotesize 1D}}{\longrightarrow} \\
%\overset{\mbox{\footnotesize 1D}}{\longrightarrow} \\
%\vdots \\
%\end{array}
%$
%&
%$
%\begin{array}{c|c|c|c|c}
% \multicolumn{1}{c}{}& \multicolumn{1}{c}{\phi_{0}^{0}(x)} & \multicolumn{1}{c}{\psi_{0}^{0}(x)} & \multicolumn{1}{c}{\psi_{0}^{1}(x)} & ... \\ \cline{2-5}
% \phi_{0}^{j}(y) & a_{0,0}' & a_{1,0}' & a_{2,0}' & \\ \cline{2-5}
% \phi_{1}^{j}(y) & a_{0,1}' & a_{1,1}' & a_{2,1}' & \\ \cline{2-5}
% \phi_{2}^{j}(y) & a_{0,2}' & a_{1,2}' & a_{2,2}' & \\ \cline{2-5}
% \vdots & & & & 
%\end{array}
%$ 
%\\[1.2cm]
%&
%&
%\settowidth\mylen{$\phi_{0}^{0}(x)$}
%$
%\begin{array}{C{\mylen} C{\mylen} C{\mylen} C{\mylen} C{\mylen}}
% & \mbox{\footnotesize 1D} \big\downarrow & \mbox{\footnotesize 1D} \big\downarrow & \mbox{\footnotesize 1D} \big\downarrow & ... \\
%\end{array}
%$
%\\[0.45cm]
%&
%&
%$
%\begin{array}{c|c|c|c|c}
% \multicolumn{1}{c}{}& \multicolumn{1}{c}{\phi_{0}^{0}(x)} & \multicolumn{1}{c}{\psi_{0}^{0}(x)} & \multicolumn{1}{c}{\psi_{0}^{1}(x)} & ... \\ \cline{2-5}
% \phi_{0}^{0}(y) & b_{0,0} & b_{1,0} & b_{2,0} & \\ \cline{2-5}
% \psi_{0}^{0}(y) & b_{0,1} & b_{1,1} & b_{2,1} & \\ \cline{2-5}
% \psi_{0}^{1}(y) & b_{0,2} & b_{1,2} & b_{2,2} & \\ \cline{2-5}
% \vdots & & & & 
%\end{array}
%$ 
%\\
%
%\end{tabular}
%\end{adjustbox}
%\end{frame}
%
%% An dieser Stelle wird an der Tafel kurz die Laufzeitkomplexität der 2D Standard-Trafo. berechnet.
%
%\begin{frame}{2D Nicht-Standard-Haar-Wavelet-Transformation}
%\begin{center}
%\textbf{2D Nicht-Standard-Basis}
%\end{center}
%~\\[-1.0cm]
%Skalierungs- und Wavelet-Funktionen sind Skalierungen von:
%\\
%\[
%\begin{array}{cc}
%\phi\phi_{0,0}^0(x,y) = \phi_0^0(x)\phi_0^0(y) & \psi\phi_{0,0}^0(x,y) = \psi_0^0(x)\phi_0^0(y) \\
%\phi\psi_{0,0}^0(x,y) = \phi_0^0(x)\psi_0^0(y) & \psi\psi_{0,0}^0(x,y) = \psi_0^0(x)\psi_0^0(y) \\
%\end{array}
%\]
%\\[0.5cm]
%
%Ziel eines Einzelschrittes:\\[0.5cm]
%
%\begin{adjustbox}{max width=\textwidth ,center}
%\begin{tabular}{c c c}
%
%$
%\begin{array}{c|c|c|c|c}
% \multicolumn{1}{c}{}& \multicolumn{1}{c}{\phi_{0}^{j}(x)} & \multicolumn{1}{c}{\phi_{1}^{j}(x)} & \multicolumn{1}{c}{\phi_{2}^{j}(x)} & ... \\ \cline{2-5}
% \phi_{0}^{j}(y) & a_{0,0} & a_{1,0} & a_{2,0} & \\ \cline{2-5}
% \phi_{1}^{j}(y) & a_{0,1} & a_{1,1} & a_{2,1} & \\ \cline{2-5}
% \phi_{2}^{j}(y) & a_{0,2} & a_{1,2} & a_{2,2} & \\ \cline{2-5}
% \vdots & & & & 
%\end{array}
%$
%&
%$\overset{?}{\rightarrow}$
%&
%$
%\begin{array}{c|c c|c c}
% \multicolumn{1}{c}{}& \multicolumn{1}{c}{\phi_{0}^{j-1}(x)} & \multicolumn{1}{c}{...} & \multicolumn{1}{c}{\psi_{0}^{j-1}(x)} & ... \\ \cline{2-5}
% \phi_{0}^{j-1}(y) & b_{0,0} & ... & b_{\frac{n}{2},0} & ...\\
% \vdots & \vdots & \ddots & \vdots & \ddots \\ \cline{2-5}
% \psi_{0}^{j-1}(y) & b_{0,\frac{n}{2}} & ... & b_{\frac{n}{2},\frac{n}{2}} & ...\\
% \vdots & \vdots & \ddots & \vdots & \ddots
%\end{array}
%$ 
%\\
%
%\end{tabular}
%\end{adjustbox}
%\end{frame}
%
%\begin{frame}{2D Nicht-Standard-Haar-Wavelet-Transformation}
%Wende 1D Einzelschritte auf Zeilen und Spalten an:
%\\[0.75cm]
%\begin{adjustbox}{max width=\textwidth ,center}
%\begin{tabular}{c c c}
%
%$
%\begin{array}{c|c|c|c|c}
% \multicolumn{1}{c}{}& \multicolumn{1}{c}{\phi_{0}^{j}(x)} & \multicolumn{1}{c}{\phi_{1}^{j}(x)} & \multicolumn{1}{c}{\phi_{2}^{j}(x)} & ... \\ \cline{2-5}
% \phi_{0}^{j}(y) & a_{0,0} & a_{1,0} & a_{2,0} & \\ \cline{2-5}
% \phi_{1}^{j}(y) & a_{0,1} & a_{1,1} & a_{2,1} & \\ \cline{2-5}
% \phi_{2}^{j}(y) & a_{0,2} & a_{1,2} & a_{2,2} & \\ \cline{2-5}
% \vdots & & & & 
%\end{array}
%$
%&
%$
%\begin{array}{c}
%\\
%\overset{\mbox{\footnotesize 1D}}{\longrightarrow} \\
%\overset{\mbox{\footnotesize 1D}}{\longrightarrow} \\
%\overset{\mbox{\footnotesize 1D}}{\longrightarrow} \\
%\vdots \\
%\end{array}
%$
%&
%$
%\begin{array}{c|c c|c c}
% \multicolumn{1}{c}{}& \multicolumn{1}{c}{\phi_{0}^{j-1}(x)} & \multicolumn{1}{c}{...} & \multicolumn{1}{c}{\psi_{0}^{j-1}(x)} & ... \\ \cline{2-5}
% \phi_{0}^{j}(y) & a_{0,0}' & ... & a_{\frac{n}{2},0}' & ...\\ \cline{2-5}
% \phi_{1}^{j}(y) & a_{0,1}' & ... & a_{\frac{n}{2},1}' & ...\\ \cline{2-5}
% \phi_{2}^{j}(y) & a_{0,2}' & ... & a_{\frac{n}{2},2}' & ...\\ \cline{2-5}
% \vdots & & & & 
%\end{array}
%$
%\\[1.2cm]
%&
%&
%\settowidth\mylen{$\phi_{0}^{0}(x)$}
%$
%\begin{array}{C{\mylen} C{\mylen} C{\mylen} C{\mylen} C{\mylen}}
% & \mbox{\footnotesize 1D} \big\downarrow & ... & \mbox{\footnotesize 1D} \big\downarrow & ... \\
%\end{array}
%$
%\\[0.45cm]
%&
%&
%$
%\begin{array}{c|c c|c c}
% \multicolumn{1}{c}{}& \multicolumn{1}{c}{\phi_{0}^{j-1}(x)} & \multicolumn{1}{c}{...} & \multicolumn{1}{c}{\psi_{0}^{j-1}(x)} & ... \\ \cline{2-5}
% \phi_{0}^{j-1}(y) & b_{0,0} & ... & b_{\frac{n}{2},0} & ...\\
% \vdots & \vdots & \ddots & \vdots & \ddots \\ \cline{2-5}
% \psi_{0}^{j-1}(y) & b_{0,\frac{n}{2}} & ... & b_{\frac{n}{2},\frac{n}{2}} & ...\\
% \vdots & \vdots & \ddots & \vdots & \ddots
%\end{array}
%$ 
%\\
%
%\end{tabular}
%\end{adjustbox}
%\end{frame}
%
%\begin{frame}[fragile]{2D Nicht-Standard-Haar-Wavelet-Transformation}
%\begin{adjustbox}{width=0.5\textwidth, center}
%\begin{tabular}{c c c}
%
%$
%\begin{array}{c|c|c|c|c}
% \multicolumn{1}{c}{}& \multicolumn{1}{c}{\phi_{0}^{j}(x)} & \multicolumn{1}{c}{\phi_{1}^{j}(x)} & \multicolumn{1}{c}{\phi_{2}^{j}(x)} & ... \\ \cline{2-5}
% \phi_{0}^{j}(y) & a_{0,0} & a_{1,0} & a_{2,0} & \\ \cline{2-5}
% \phi_{1}^{j}(y) & a_{0,1} & a_{1,1} & a_{2,1} & \\ \cline{2-5}
% \phi_{2}^{j}(y) & a_{0,2} & a_{1,2} & a_{2,2} & \\ \cline{2-5}
% \vdots & & & & 
%\end{array}
%$
%&
%$
%\begin{array}{c}
%\\
%\overset{\mbox{\footnotesize 1D}}{\longrightarrow} \\
%\overset{\mbox{\footnotesize 1D}}{\longrightarrow} \\
%\overset{\mbox{\footnotesize 1D}}{\longrightarrow} \\
%\vdots \\
%\end{array}
%$
%&
%$
%\begin{array}{c|c c|c c}
% \multicolumn{1}{c}{}& \multicolumn{1}{c}{\phi_{0}^{j-1}(x)} & \multicolumn{1}{c}{...} & \multicolumn{1}{c}{\psi_{0}^{j-1}(x)} & ... \\ \cline{2-5}
% \phi_{0}^{j}(y) & a_{0,0}' & ... & a_{\frac{n}{2},0}' & ...\\ \cline{2-5}
% \phi_{1}^{j}(y) & a_{0,1}' & ... & a_{\frac{n}{2},1}' & ...\\ \cline{2-5}
% \phi_{2}^{j}(y) & a_{0,2}' & ... & a_{\frac{n}{2},2}' & ...\\ \cline{2-5}
% \vdots & & & & 
%\end{array}
%$
%\\[1.2cm]
%&
%&
%\settowidth\mylen{$\phi_{0}^{0}(x)$}
%$
%\begin{array}{C{\mylen} C{\mylen} C{\mylen} C{\mylen} C{\mylen}}
% & \mbox{\footnotesize 1D} \big\downarrow & ... & \mbox{\footnotesize 1D} \big\downarrow & ... \\
%\end{array}
%$
%\\[0.45cm]
%&
%&
%$
%\begin{array}{c|c c|c c}
% \multicolumn{1}{c}{}& \multicolumn{1}{c}{\phi_{0}^{j-1}(x)} & \multicolumn{1}{c}{...} & \multicolumn{1}{c}{\psi_{0}^{j-1}(x)} & ... \\ \cline{2-5}
% \phi_{0}^{j-1}(y) & b_{0,0} & ... & b_{\frac{n}{2},0} & ...\\
% \vdots & \vdots & \ddots & \vdots & \ddots \\ \cline{2-5}
% \psi_{0}^{j-1}(y) & b_{0,\frac{n}{2}} & ... & b_{\frac{n}{2},\frac{n}{2}} & ...\\
% \vdots & \vdots & \ddots & \vdots & \ddots
%\end{array}
%$ 
%\\
%
%\end{tabular}
%\end{adjustbox}
%\center
%\begin{adjustbox}{max width=0.95\linewidth, fbox}
%\begin{adjustbox}{minipage=\linewidth, scale=0.8, center}
%\begin{verbatim}
%def nonstandardAnalysis(C):
%    C' = copy(C)
%    h = C.size
%    while h > 1:
%        for i in range(h):
%            C'[i,0:h] = analysisStep(C'[i,0:h])
%        for j in range(h):
%            C'[0:h,j] = analysisStep(C'[0:h,j])
%        h = h // 2
%    return C'
%\end{verbatim}
%\end{adjustbox}
%\end{adjustbox}
%
%\end{frame}
%
%% An dieser Stelle führe noch die Laufzeitkomplexitätsanalyse für die Nicht-Standard-Transformation durch und Leite über Richtung B-Splines.

}

\section{Splinekurven} {

\begin{frame}{Grundlagen: Splinefunktionen}
\textbf{Definition:} Splinefunktion \\
Sei $\mathbf{t}=(t_0,...,t_n)$ eine monoton steigende Sequenz reeller Zahlen und $d \in \N_0$. Eine Funktion $S:[t_0,t_n] \rightarrow \R$ in $C^{d-1}(\R)$ heißt Splinefunktion vom Grad $d$ zum Knotenvektor $\mathbf{t}$, wenn $S|_{[t_{k},t_{k+1}]}$, $k=0,...,n-1$, ein Polynom vom Maximalgrad $d$ ist. 
\\[1.0cm]

\pause
Die Menge aller Splinefunktionen selben Grades $d$ und zum selben Knotenvektor $\mathbf{t}$ bilden einen reellen Vektorraum mit $n+d$ Dimensionen. 
\\
\end{frame}

\begin{frame}{Grundlagen: B-Spline-Funktionen}
\textbf{Definition:} B-Spline-Funktionen \\
Sei $\mathbf{\tau}=(t_{-d},...,t_0,...,t_n,...t_{n+d})$ eine monoton steigende Sequenz reeller Zahlen und $d \in \N_0$. Definiere die B-Spline-Funktionen $(B_{i,d})_{i=1}^{n+d}$ vom Grad $d$ zu den Knoten $\mathbf{\tau}$ durch 
\[
B_{i,0}(x) := \begin{cases} 1& \tau_i<x<\tau_{i+1} \\ 0& \text{sonst} \end{cases} \qquad ,
\]
\[
B_{i,k}(x) := \frac{x-\tau_i}{\tau_{i+k}-\tau_i}B_{i,k-1}(x) + \frac{\tau_{i+k+1}-x}{\tau_{i+k+1}-\tau_{i+1}}B_{i+1,k-1}(x)
.
\]
\\[1.0cm]

$(B_{i,d})_{i=1}^{n+d}$ ist eine Basis des Splineraums vom Grad $d$ zum Knotenvektor $\mathbf{t}=(t_0,...,t_n)$.
\\
\end{frame}

\begin{frame}{Grundlagen: B-Spline-Kurven}
\begin{adjustbox}{max width=\textwidth, center}
\begin{tabular}{c c}
\begin{minipage}{0.45\textwidth}
\textbf{Definition:} B-Spline-Kurve \\
Für $(\mathbf{c}_i)_{i=1}^{n+d}$, $\mathbf{c}_i \in \R^m$ heißt
\[
\mathbf{f}(x):=\sum_{i=1}^{n+d}\mathbf{c_i}B_{i,d}(x)
\]
B-Spline-Kurve zu den \\ Kontrollpunkten $(\mathbf{c}_i)_{i=1}^{n+d}$.
\end{minipage}
&
\begin{minipage}{0.5\textwidth}
\includegraphics[width=\textwidth]{b_spline_curve.pdf}
\end{minipage}

\end{tabular}
\end{adjustbox}
\end{frame}

\begin{frame}{Multiskalenanalyse von Splinekurven}
Analyse von B-Spline-Kurven ergibt sich Komponentenweise.
\\[1.0cm]
Kochrezept:
\begin{itemize}
\item
Geschachtelte Splineräume: \\
Die Splineräume $V_d^k$ eines Grades $d$ zu den Knoten
\[
\mathbf{t}^k=(\frac{0}{2^k},\frac{1}{2^k},...,\frac{2^k-1}{2^k},\frac{2^k}{2^k})
\]
sind geschachtelt. Wähle als Skalenfunktionen die B-Splines zu
\[
\mathbf{\tau}^k=(\underbrace{0,...,0}_{d},\frac{0}{2^k},\frac{1}{2^k},...,\frac{2^k-1}{2^k},\frac{2^k}{2^k},\underbrace{1,...,1}_{d})
.
\]
\end{itemize}
\end{frame}


% Splitte Folie vermutlich etwa hier...

\begin{frame}{Multiskalenanalyse von Splinekurven}
\begin{itemize}
\item Standard-Skalarprodukt:
\[
\langle f,g\rangle =\int_0^1f(x)g(x)dx
\]
\item Wavelets:\\
Wahl ist nicht eindeutig! Man verlangt zum Beispiel minimalen Träger, um die Wavelets festzulegen. Diese beschleunigt zudem die Berechnungen.
\end{itemize}
\end{frame}

}




%\section{Anwendungen}
%	
%	{\setbeamertemplate{frame footer}{Quelle}
%	\begin{frame}{Titel}
%		\begin{columns}[T,onlytextwidth]
%			\column{0.75\textwidth}
%			Inhalt
%			\column{0.25\textwidth}
%			Inhalt 2
%		\end{columns}
%		Inhalt 3 \pause
%		\begin{columns}[c,onlytextwidth]
%			\column{0.38\textwidth}
%			\centering
%			Inhalt 4
%			
%			\column{0.61\textwidth}
%			\vspace{15pt}\ \\
%			\centering {\large$\Rightarrow$ Inhalt 5}
%		\end{columns}
%	\end{frame}}